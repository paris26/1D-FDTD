\documentclass[11pt]{article}
\usepackage{amsmath, amssymb}
\usepackage{geometry}
\geometry{margin=1in}
\usepackage{graphicx}

\title{Introduction to 1D FDTD}
\author{}
\date{}

\begin{document}
\maketitle

\section{Maxwell's Equations in 1D}

In one dimension, with the electric field $E$ polarized in the $x$-direction and the wave traveling in the $z$-direction, Maxwell's equations reduce to two coupled partial differential equations:

\begin{align}
\frac{\partial E}{\partial z} &= -\mu_0 \frac{\partial H}{\partial t} \label{eq:faraday}\\[8pt]
\frac{\partial H}{\partial z} &= -\varepsilon_0 \frac{\partial E}{\partial t} \label{eq:ampere}
\end{align}

\textbf{Physical interpretation:}
\begin{itemize}
    \item Equation \eqref{eq:faraday}: A spatial change in $E$ creates a time change in $H$ (Faraday's law)
    \item Equation \eqref{eq:ampere}: A spatial change in $H$ creates a time change in $E$ (Amp\`ere's law)
\end{itemize}

These fields ``chase each other'' through space, producing a wave traveling at speed:
\[
c = \frac{1}{\sqrt{\mu_0 \varepsilon_0}} \approx 3 \times 10^8 \text{ m/s}
\]

\section{The FDTD Idea: Discretizing Space and Time}

FDTD (Finite-Difference Time-Domain) replaces continuous derivatives with finite differences. The key insight is the \textbf{Yee scheme}: stagger $E$ and $H$ in both space and time.

\begin{center}
\begin{tabular}{c}
\textbf{Staggered Grid Layout} \\[6pt]
\begin{tabular}{ccccccc}
$E_{k-1}$ & & $H_{k-\frac{1}{2}}$ & & $E_k$ & & $H_{k+\frac{1}{2}}$ \\
$|$ & & $|$ & & $|$ & & $|$ \\
\hline
\multicolumn{7}{c}{$\xleftarrow{\hspace{1cm}} \Delta z \xrightarrow{\hspace{1cm}}$}
\end{tabular}
\end{tabular}
\end{center}

Similarly in time: $E$ is computed at integer time steps ($n$), and $H$ at half-integer steps ($n + \frac{1}{2}$).

\textbf{Why stagger?} Centered differences are second-order accurate. If $E$ and $H$ are at the same points, you'd need one-sided differences (first-order, less accurate).

\section{The Update Equations}

Discretize equation \eqref{eq:faraday} at position $k + \frac{1}{2}$ and time $n$:
\[
\frac{E^n_{k+1} - E^n_k}{\Delta z} = -\mu_0 \frac{H^{n+\frac{1}{2}}_{k+\frac{1}{2}} - H^{n-\frac{1}{2}}_{k+\frac{1}{2}}}{\Delta t}
\]

Solve for the new $H$:
\begin{equation}
\boxed{H^{n+\frac{1}{2}}_{k+\frac{1}{2}} = H^{n-\frac{1}{2}}_{k+\frac{1}{2}} - \frac{\Delta t}{\mu_0 \Delta z}\left(E^n_{k+1} - E^n_k\right)}
\label{eq:H_update}
\end{equation}

Similarly, discretize equation \eqref{eq:ampere} at position $k$ and time $n + \frac{1}{2}$:
\begin{equation}
\boxed{E^{n+1}_k = E^n_k - \frac{\Delta t}{\varepsilon_0 \Delta z}\left(H^{n+\frac{1}{2}}_{k+\frac{1}{2}} - H^{n+\frac{1}{2}}_{k-\frac{1}{2}}\right)}
\label{eq:E_update}
\end{equation}

\textbf{The algorithm} (leap-frog time stepping):
\begin{enumerate}
    \item Update all $H$ values using current $E$ values (Eq.~\ref{eq:H_update})
    \item Update all $E$ values using new $H$ values (Eq.~\ref{eq:E_update})
    \item Repeat for each time step
\end{enumerate}

\section{Stability: The CFL Condition}

The algorithm is only stable if information doesn't travel faster numerically than physically. This requires:
\begin{equation}
\boxed{\Delta t < \frac{\Delta z}{c}}
\label{eq:cfl}
\end{equation}

This is the \textbf{Courant-Friedrichs-Lewy (CFL) condition}.

\textbf{Intuition:} In one time step $\Delta t$, the wave travels distance $c \cdot \Delta t$. This must be less than one grid cell $\Delta z$, or the wave ``skips'' cells and the simulation blows up.

A safe choice is $\Delta t = \frac{\Delta z}{2c}$ (Courant number = 0.5).

\section{Physical Constants}

For reference:
\begin{align*}
\varepsilon_0 &= 8.854 \times 10^{-12} \text{ F/m} \quad \text{(permittivity of free space)}\\
\mu_0 &= 4\pi \times 10^{-7} \text{ H/m} \quad \text{(permeability of free space)}\\
c &= 2.998 \times 10^8 \text{ m/s} \quad \text{(speed of light)}
\end{align*}

\section{Next Steps}

Once this vacuum code works:
\begin{enumerate}
    \item Add a dielectric region: replace $\varepsilon_0 \to \varepsilon_0 \varepsilon_r$ in the $E$ update
    \item Add dispersion using the ADE (Auxiliary Differential Equation) method
    \item Validate against analytical Fresnel coefficients
\end{enumerate}

\end{document}

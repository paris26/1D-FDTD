% Compile with XeLaTeX (recommended for Greek Unicode)
\documentclass[11pt,a4paper]{article}

\usepackage{fontspec}
\usepackage{polyglossia}
\setmainlanguage{greek}
\setotherlanguage{english}
\setmainfont{Noto Serif}
\newfontfamily\greekfont{Noto Serif}

\usepackage{amsmath,amssymb}
\usepackage{geometry}
\geometry{margin=2.2cm}
\usepackage{microtype}
\usepackage{hyperref}

\title{Σημειώσεις (Feynman, Vol.~I, Κεφ.~28--33):\\
Από τις εξισώσεις Maxwell σε κύματα, υλικά, και «γέφυρα» προς FDTD}
\author{}
\date{}

\begin{document}
\maketitle

\vspace{-1.2em}
\noindent
\textbf{Στόχος.} Να κρατήσω μόνο τα \emph{pivotal} βήματα/εξισώσεις που χρησιμοποιώ σαν νοητικό «σκελετό»:
(α) πώς προκύπτει η εικόνα των ηλεκτρομαγνητικών κυμάτων (Maxwell), 
(β) πώς ο Feynman στο Κεφ.~28--29 περνάει σε \emph{ακτινοβολία} με καθυστέρηση (retardation),
(γ) πώς η \emph{συμβολή/περίθλαση} είναι απλώς άθροισμα φάσεων,
(δ) πώς το \emph{υλικό} δίνει δείκτη διάθλασης μέσω \emph{ταλαντωτών} (Lorentz model),
(ε) πώς μετράμε \emph{ενέργεια/ισχύ} (Poynting/Larmor),
(στ) γιατί η \emph{πόλωση} είναι αναγκαστικά διανυσματικό παιχνίδι.
Στο τέλος γράφω 1 σελίδα «μετάφραση» προς FDTD.

\section{Το minimum από Maxwell που θέλω για FDTD (και το σχόλιο του Feynman)}
Ο Feynman στο Κεφ.~28 υπενθυμίζει ότι ο Maxwell είδε ασυνέπειες στους τότε νόμους και \emph{έπρεπε να προσθέσει έναν όρο} (η «διόρθωση» που σήμερα λέμε \emph{displacement current}). Αυτό είναι το κομμάτι που κάνει τα κύματα/το φως να πέφτουν σαν $1/r$ και όχι σαν $1/r^2$.

Για το FDTD παίρνω τις εξισώσεις Maxwell σε διαφορική μορφή (SI):
\begin{align}
\nabla\cdot \mathbf{E} &= \frac{\rho}{\varepsilon_0}, &
\nabla\cdot \mathbf{B} &= 0, \\
\nabla\times \mathbf{E} &= -\frac{\partial \mathbf{B}}{\partial t}, &
\nabla\times \mathbf{B} &= \mu_0\mathbf{J}+\mu_0\varepsilon_0\frac{\partial \mathbf{E}}{\partial t}. 
\end{align}
Η κρίσιμη «Maxwell-προσθήκη» είναι ο όρος $\mu_0\varepsilon_0 \,\partial \mathbf{E}/\partial t$.

\paragraph{Το pivot derivation: κύμα στο κενό.}
Στο κενό ($\rho=0$, $\mathbf{J}=0$):
\[
\nabla\times(\nabla\times\mathbf{E})=-\frac{\partial}{\partial t}(\nabla\times\mathbf{B})
=-\mu_0\varepsilon_0\frac{\partial^2\mathbf{E}}{\partial t^2}.
\]
Με $\nabla\times(\nabla\times\mathbf{E})=\nabla(\nabla\cdot\mathbf{E})-\nabla^2\mathbf{E}$ και $\nabla\cdot\mathbf{E}=0$:
\[
\nabla^2\mathbf{E}=\mu_0\varepsilon_0\frac{\partial^2\mathbf{E}}{\partial t^2}
\quad\Rightarrow\quad
\frac{\partial^2\mathbf{E}}{\partial t^2}=c^2\nabla^2\mathbf{E},
\qquad c=\frac{1}{\sqrt{\mu_0\varepsilon_0}}.
\]
(Το ίδιο και για $\mathbf{B}$.)
\emph{Αυτό} είναι το PDE που εγώ «βλέπω» πίσω από κάθε FDTD update.

\section{Κεφ. 28--29: Ακτινοβολία και καθυστέρηση (retarded time)}
\subsection{Η βασική ιδέα: ό,τι βλέπεις τώρα προέρχεται από το παρελθόν}
Ο Feynman το λέει με μια απλή εικόνα:
\emph{«Σαν το φορτίο να κουβαλάει ένα φως»} --- άρα δεν «βλέπεις» πού είναι τώρα, αλλά πού \emph{ήταν} όταν έφυγε το σήμα, με καθυστέρηση $r/c$.

\subsection{Το pivot αποτέλεσμα: το $1/r$ κομμάτι (radiation term)}
Στο Κεφ.~28 γράφει μια (συμπυκνωμένη) έκφραση για $\mathbf{E}$ από κινούμενο φορτίο που έχει 3 κομμάτια: 
\emph{Coulomb} ($\sim 1/r^2$), μια \emph{διόρθωση καθυστέρησης} ($\sim 1/r^2$), 
και το \emph{ακτινοβολούμενο} μέρος ($\sim 1/r$).
Το κρίσιμο είναι ότι, μακριά, κρατάμε μόνο το τρίτο:
\begin{equation}
\mathbf{E}_{\text{rad}}\;\propto\; \frac{1}{c^2}\frac{d^2\hat{\mathbf{r}}'}{dt^2}\;\;\Longrightarrow\;\;
E_x(t)=\frac{-q}{4\pi\varepsilon_0 c^2\,r}\,a_x\!\left(t-\frac{r}{c}\right).
\end{equation}
(Αυτό είναι η «καθαρή» μορφή που ο Feynman παίρνει ως νόμο για μη-σχετικιστική κίνηση, σε μεγάλη απόσταση.)

Στο Κεφ.~29 το γράφει και με γωνιακή εξάρτηση (dipole pattern):
\begin{equation}
E(t)=\frac{-q\,a(t-r/c)\,\sin\theta}{4\pi\varepsilon_0 c^2\,r}.
\end{equation}

\subsection{Αναλογία-κλειδί που χρησιμοποιεί: «snapshot = αντεστραμμένο» $a(t)$}
Στο Κεφ.~29 δείχνει ότι $a(t-r/c)$ σημαίνει:
\begin{itemize}
\item σε κάθε σημείο, το πεδίο «γράφει» την επιτάχυνση σε \emph{προηγούμενο} χρόνο,
\item αν κοιτάξεις σε μια \emph{στιγμή} το $E$ ως συνάρτηση του $r$, παίρνεις ουσιαστικά ένα «αντεστραμμένο» γράφημα του $a(t)$,
\item το μοτίβο μετακινείται προς τα έξω με ταχύτητα $c$.
\end{itemize}
Μαθηματικά: για φάση $\phi=\omega(t-r/c)$, ο κυματάριθμος είναι
\begin{equation}
k=\left|\frac{\partial \phi}{\partial r}\right|=\frac{\omega}{c},
\qquad \lambda=\frac{2\pi}{k}.
\end{equation}
Για FDTD αυτό μεταφράζεται απλά ως: \emph{η πληροφορία ταξιδεύει με $c$ και η φάση «μετριέται» στον χώρο.}

\section{Κεφ. 29--30: Συμβολή/Περίθλαση = άθροισμα (πολλών) φάσεων}
\subsection{Το pivot: προσθέτεις πλάτη, μετράς ένταση}
Ο Feynman τονίζει επανειλημμένα ότι αυτό που «φαίνεται» (ισχύς/ένταση) είναι ανάλογο του τετραγώνου:
\[
I \propto \langle E^2\rangle.
\]
Όταν αθροίζεις δύο συνεισφορές (σύνθετα πλάτη) παίρνεις:
\begin{equation}
A_R^2=A_1^2+A_2^2+2A_1A_2\cos(\phi_1-\phi_2).
\end{equation}
(Το \emph{cross term} είναι όλη η συμβολή.)

\subsection{Κεφ.~30: $n$ ίσοι ταλαντωτές (grating) και το γεωμετρικό «πολύγωνο»}
Το βασικό άθροισμα που γράφει είναι:
\begin{equation}
R=A\sum_{s=0}^{n-1}\cos(\omega t+s\phi),
\qquad
\phi=\alpha+\frac{2\pi d\sin\theta}{\lambda}.
\end{equation}
Η ωραία ιδέα είναι ότι αυτό είναι \emph{πολύγωνο} στο μιγαδικό επίπεδο (ίσα διανύσματα με σταθερή γωνία μεταξύ τους).

Ένα απλό, χρήσιμο αποτέλεσμα για «ικανότητα διαχωρισμού» (resolving power) φράγματος:
\begin{equation}
\frac{\Delta\lambda}{\lambda}=\frac{1}{mn}.
\end{equation}

\subsection{Η «παράξενη» αλλά πρακτική αρχή οπών}
Στην περίθλαση από οπές/σκιά, ο Feynman λέει ότι παίρνεις πολύ καλή προσέγγιση αν:
\begin{quote}
θεωρήσεις τις \textbf{οπές} σαν να είναι οι «πηγές» με φάσεις όπως θα ήταν \emph{αν δεν υπήρχε} το αδιαφανές.
\end{quote}
Αυτό είναι χρήσιμο ως mental model: \emph{η γεωμετρία του ανοίγματος μπαίνει σαν χωρική κατανομή πηγών.}

\section{Κεφ. 31: Δείκτης διάθλασης από «άτομα-ταλαντωτές» (Lorentz)}
\subsection{Το pivot: διάθλαση = καθυστέρηση φάσης}
Για μια λεπτή πλάκα πάχους $\Delta z$ ο Feynman γράφει το «όλο αποτέλεσμα» ως καθυστέρηση:
\begin{equation}
E_{\text{after}}=e^{-i\omega(n-1)\Delta z/c}\,E_s
\approx \left(1-i\omega\frac{(n-1)\Delta z}{c}\right)E_s
=E_s+E_a.
\end{equation}
Δηλαδή: \textbf{το υλικό προσθέτει ένα μικρό πεδίο} $E_a$ σχεδόν κάθετο (σε φάση) στο $E_s$.

\subsection{Το pivot μοντέλο: ο ηλεκτρόνιος σαν ελατήριο}
Το άτομο μοντελοποιείται σαν γραμμικός ταλαντωτής:
\begin{equation}
m\left(\ddot{x}+\omega_0^2 x\right)=q_e E_0 e^{i\omega t}
\quad\Rightarrow\quad
x=\frac{q_eE_0}{m(\omega_0^2-\omega^2)}e^{i\omega t}.
\end{equation}

\subsection{Το κύριο αποτέλεσμα: $n(\omega)$}
Στο τέλος παίρνει:
\begin{equation}
n=1+\frac{Nq_e^2}{2\varepsilon_0\,m(\omega_0^2-\omega^2)}.
\end{equation}
\textbf{Για FDTD:} αυτό είναι πρακτικά το ίδιο με \emph{Lorentz dispersive medium}. 
Αν θες να το βάλεις στον χρόνο, κρατάς μια δυναμική για την πόλωση $P$ (Auxiliary Differential Equation)
της μορφής $\ddot{P}+\omega_0^2 P \propto E$.

\section{Κεφ. 32: Ενέργεια, Poynting, Larmor, damping, σκέδαση}
\subsection{Ενέργεια ανά μονάδα επιφάνειας: «η αντίσταση του κενού»}
Ο Feynman γράφει ότι το ενεργειακό flux είναι $\varepsilon_0 c E^2$ και ότι
\[
\frac{1}{\varepsilon_0 c}=377~\Omega,
\]
άρα (για κυματομορφές) η ισχύς/εμβαδόν είναι \emph{μέσο} $E^2$ διαιρεμένο με $377$.

\subsection{Larmor (ολική ακτινοβολούμενη ισχύς)}
Για επιταχυνόμενο φορτίο, η ροή ισχύος ανά $\text{m}^2$ στη διεύθυνση $\theta$:
\begin{equation}
S(\theta)=\frac{q^2 a'^2\sin^2\theta}{16\pi^2\varepsilon_0 r^2 c^3}.
\end{equation}
Ολοκληρώνοντας σε όλες τις γωνίες:
\begin{equation}
P=\frac{q^2 a'^2}{6\pi\varepsilon_0 c^3}.
\end{equation}
\textbf{Για FDTD:} η αντίστοιχη «μετρήσιμη» ποσότητα είναι $\mathbf{S}=\mathbf{E}\times\mathbf{H}$ και ολοκλήρωση σε κλειστή επιφάνεια.

\subsection{Radiation damping (η απώλεια κάνει το ταλαντωτή να σβήνει)}
Ο Feynman ορίζει $Q$ ως
\[
Q=-\frac{\omega W}{dW/dt},
\]
και δείχνει πως ακόμη και χωρίς «τριβή», ένα φορτισμένο ελατήριο σβήνει επειδή \emph{ακτινοβολεί}.

\subsection{Σκέδαση: γιατί συχνά δεν βλέπεις συμβολή}
Με πολλούς τυχαίους σκεδαστές, οι φάσεις αλλάζουν «άναρχα» και τα cross terms
κατά μέσο όρο μηδενίζονται, άρα μένει περίπου άθροισμα εντάσεων.

\section{Κεφ. 33: Πόλωση = το $\mathbf{E}$ είναι \emph{διάνυσμα}}
\subsection{Το pivot: δύο κάθετες ταλαντώσεις $\Rightarrow$ έλλειψη}
Ο Feynman ξεκινάει από υπέρθεση δύο κάθετων συνιστωσών ίδιας συχνότητας:
\[
E_x=A\cos(\omega t),\qquad
E_y=B\cos(\omega t+\delta).
\]
\begin{itemize}
\item $\delta=0$ (ή $\pi$): γραμμική πόλωση.
\item $\delta=\pm\pi/2$ και $A=B$: κυκλική πόλωση.
\item γενικά: ελλειπτική πόλωση.
\end{itemize}
Η αναλογία που χρησιμοποιεί για την έλλειψη είναι \textbf{ένα εκκρεμές/μπάλα σε νήμα} που μπορεί να ταλαντώνεται σε $x$ και $y$ με ίδια συχνότητα --- αν οι φάσεις δεν ταιριάζουν, η τροχιά γίνεται έλλειψη.

\subsection{Unpolarized light (πρακτικά) = γρήγορα μεταβαλλόμενη πόλωση}
Αν η πόλωση αλλάζει ταχύτερα απ' όσο μπορείς να τη «δειγματοληπτήσεις», τότε οι πολωτικοί όροι κατά μέσο όρο σβήνουν και το φως λέγεται μη-πολωμένο.

\subsection{Διπλοθλαστικότητα/πλάκες $\lambda/2$: διαφορετικές ταχύτητες για διαφορετικούς άξονες}
Στο παράδειγμα με cellophane/polaroids, το υλικό συμπεριφέρεται σαν να έχει δύο ταχύτητες (δύο δείκτες) για δύο κάθετες διευθύνσεις πόλωσης.
Μια \emph{half-wave} πλάκα μπορεί να περιστρέψει γραμμική πόλωση κατά $90^\circ$ όταν η είσοδος είναι στα $45^\circ$ ως προς τον οπτικό άξονα.

\section{Μίνι «μετάφραση» προς FDTD (πώς θα το έστηνα σαν πρόβλημα)}
\subsection*{1) Ποιο PDE λύνω;}
Λύνω τα curl-Maxwell στον χρόνο:
\[
\frac{\partial \mathbf{B}}{\partial t}=-\nabla\times\mathbf{E},\qquad
\frac{\partial \mathbf{D}}{\partial t}= \nabla\times\mathbf{H}-\mathbf{J},
\]
με $\mathbf{B}=\mu\mathbf{H}$ και $\mathbf{D}=\varepsilon\mathbf{E}$ (ή διασπορά με $P(t)$).

\subsection*{2) Το update mental model (leapfrog)}
Σκέφτομαι ότι \textbf{η μεταβολή του $\mathbf{E}$ «γεννά» $\mathbf{H}$ και αντίστροφα}. 
Άρα χρονικά τα ενημερώνω εναλλάξ (leapfrog).

\subsection*{3) Courant stability (ένα αριθμητικό παράδειγμα)}
Για 3D ο περιορισμός είναι περίπου:
\[
\Delta t \le \frac{1}{c\sqrt{\frac{1}{\Delta x^2}+\frac{1}{\Delta y^2}+\frac{1}{\Delta z^2}}}.
\]
\textbf{Παράδειγμα:} αν $\Delta x=\Delta y=\Delta z=1~\text{mm}$,
τότε $\Delta t_{\max}\approx \frac{10^{-3}}{(3\times10^8)\sqrt{3}}
\approx 1.9\times10^{-12}\ \text{s}$.

\subsection*{4) Πηγές και παρατηρήσιμα}
\begin{itemize}
\item \textbf{Πηγή dipole:} ρεύμα $\mathbf{J}(t)$ σε ένα κελί $\Rightarrow$ εκπέμπει, σαν το Κεφ.~28--29.
\item \textbf{Μετρήσεις:} $\mathbf{S}=\mathbf{E}\times\mathbf{H}$, ολοκλήρωση σε επιφάνεια $\Rightarrow$ ισχύς (Κεφ.~32).
\item \textbf{Συμβολή/περίθλαση:} παίρνω πεδία σε πολλαπλά σημεία και φτιάχνω pattern από $|E|^2$ (Κεφ.~29--30).
\item \textbf{Υλικά:} σταθερό $n$ $\Rightarrow$ $\varepsilon_r=n^2$· διασπορά $\Rightarrow$ Lorentz ADE από το Κεφ.~31.
\item \textbf{Πόλωση/ανισοτροπία:} κρατάω τα 3 components και (αν χρειαστεί) tensor $\varepsilon$ (Κεφ.~33).
\end{itemize}

\vspace{0.5em}
\noindent
\textbf{Κλείσιμο:} Αν κάποιος καταλάβει αυτά τα pivots, τότε το FDTD γίνεται απλώς \emph{μηχανική}:
διακριτοποίηση των curl-Maxwell + σωστά $\Delta t$ + σωστά υλικά + σωστή μέτρηση ισχύος/μοτίβων.

\end{document}
